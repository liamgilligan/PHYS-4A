\documentclass[11pt, letterpaper, includehead]{article}

%%%%%%%%%%%%%%%%%%%%% Pre-document %%%%%%%%%%%%%%%%%%%%%
\usepackage{fancyhdr}  % Allow for headers
\usepackage{graphicx}  % Allow for figures 
\usepackage{float}     % Allow for figure inserted in specified location
\usepackage{amsmath}   % Allow for aligned math
\usepackage{array}     % Allow for cell width manipulation
\usepackage[dvipsnames]{xcolor}    % Allow for colors + more colors!
\usepackage{amssymb} % Uhhhh what was this????

\setlength{\parindent}{0pt} % Remove auto paragraph indents

% Get rid of those big ass margins
\usepackage[margin=1in]{geometry}

% Table cell formatting
\setlength{\arrayrulewidth}{0.25mm}
\setlength{\tabcolsep}{11pt}
\renewcommand{\arraystretch}{1.2}

\begin{document}

%%%%%%%%%%%%%%%%%%%%% Title Page %%%%%%%%%%%%%%%%%%%%%
\begin{titlepage}
  \begin{center}
    \Huge{\textbf{Lab 2}}\\
    \Huge{Experimental Uncertainty}
    \vfill
    \large{\textbf{Rectangle Repulsed Researchers}}\\
    \large{Julian Barossi, Liam Gilligan, Stephanie L'Heureux}\\
    \vspace{0.5cm}
    \normalsize
    \today
  \end{center}
\end{titlepage}

%%%%%%%%%%%%%%%%%%%%% TABLE OF CONTENTS %%%%%%%%%%%%%%%%%%%%%
\tableofcontents
\pagebreak % Move to next page

% Add a nice fancy header
\pagestyle{fancy}
\fancyhead{}
\fancyhead[C]{\textbf{Lab 2:} Experimental Uncertainty}

\section{Measuring the Time of a Dropped Pencil} % 1

\subsection{Predicted time $t_{thy}$} % 1.1
The calculations below predict the time ($t_{thy}$) for the pencil to
fall from rest a distance of $2.0m$.

$$y       = y_0 + v_{y_0}t + \frac{1}{2}a_yt^2$$
$$-y_0    = \frac{1}{2}a_yt^2$$
$$t       = \sqrt{\frac{-2y_0}{a_y}}$$
$$t       = \sqrt{\frac{-2(2.0m)}{-9.8m/s^2}}$$
$$\boxed{t_{thy}\approx 0.64s}$$

\renewcommand{\arraystretch}{1.2}
\subsection{Recorded time} % 1.2 
\begin{center}
  \begin{tabular}[H]{| m{2cm} | m{2cm} |}
    \hline
    \textbf{Trial} & \textbf{Time (s)} \\ [0.5ex]
    \hline \hline
    1              & 0.55              \\
    \hline
    2              & 0.69              \\
    \hline
    3              & 0.69              \\
    \hline
    4              & 0.65              \\
    \hline
    5              & 0.67              \\
    \hline
    6              & 0.58              \\
    \hline
    7              & 0.61              \\
    \hline
    8              & 0.64              \\
    \hline
    9              & 0.63              \\
    \hline
    10             & 0.68              \\
    \hline
    11             & 0.61              \\
    \hline
    12             & 0.63              \\
    \hline
    13             & 0.67              \\
    \hline
    14             & 0.64              \\
    \hline
    15             & 0.65              \\
    \hline
    16             & 0.68              \\
    \hline
    17             & 0.66              \\
    \hline
    18             & 0.66              \\
    \hline
    19             & 0.56              \\
    \hline
    20             & 0.66              \\
    \hline
  \end{tabular}
\end{center}

\setcounter{subsection}{3} % Make next section start at 1.4
\subsection{Mean average time for 100 measurements} % 1.4

\begin{center}
  \begin{tabular}{|  c | c | c | c | c | c | }
    \hline
    \textbf{Kate} & \textbf{Jesus} & \textbf{Sam} & \textbf{Steph} & \textbf{Liam} \\
    \hline\hline
    0.67          & 0.53           & 0.68         & 0.55           & 0.66          \\
    \hline
    0.57          & 0.66           & 0.71         & 0.69           & 0.68          \\
    \hline
    0.51          & 0.61           & 0.64         & 0.69           & 0.66          \\
    \hline
    0.56          & 0.52           & 0.61         & 0.65           & 0.60          \\
    \hline
    0.55          & 0.61           & 0.63         & 0.67           & 0.55          \\
    \hline
    0.59          & 0.63           & 0.70         & 0.58           & 0.55          \\
    \hline
    0.63          & 0.61           & 0.64         & 0.61           & 0.60          \\
    \hline
    0.63          & 0.64           & 0.72         & 0.64           & 0.56          \\
    \hline
    0.63          & 0.67           & 0.65         & 0.63           & 0.61          \\
    \hline
    0.76          & 0.69           & 0.62         & 0.68           & 0.63          \\
    \hline
    0.56          & 0.53           & 0.61         & 0.61           & 0.66          \\
    \hline
    0.70          & 0.56           & 0.60         & 0.63           & 0.68          \\
    \hline
    0.66          & 0.59           & 0.64         & 0.67           & 0.68          \\
    \hline
    0.74          & 0.54           & 0.62         & 0.64           & 0.70          \\
    \hline
    0.67          & 0.64           & 0.66         & 0.65           & 0.63          \\
    \hline
    0.71          & 0.61           & 0.67         & 0.68           & 0.63          \\
    \hline
    0.75          & 0.57           & 0.65         & 0.66           & 0.65          \\
    \hline
    0.58          & 0.63           & 0.59         & 0.66           & 0.70          \\
    \hline
    0.70          & 0.56           & 0.64         & 0.56           & 0.61          \\
    \hline
    0.60          & 0.64           & 0.64         & 0.66           & 0.60          \\
    \hline
    \hline
    \multicolumn{5}{|c|}{\textbf{Average}} \\
    \hline
    0.639	& 0.602	& 0.646	& 0.641	& 0.632 \\
    \hline
  \end{tabular}
\end{center}


  $$\bar{t} = \frac{1}{n}\sum t_i$$\\
  $$\bar{t} = 0.6318s$$

\subsection{Standard deviation for 100 measurements} % 1.5


  $$\sigma = \sqrt{\frac{1}{N - 1}\sum_{i = 1}^{N} (d_i)^2} = \sqrt{\frac{1}{N - 1}\sum_{i = 1}^{N} (t_i - \bar{t})^2}$$\\
  $$\sigma = \sqrt{\frac{1}{100 - 1}\sum_{i = 1}^{100}(t_i - 0.6318s)^2}$$\\
  $$\sigma = 0.05224998188 \approx 0.05$$


\subsection{Values within one standard deviation} % 1.6 
Approximately $68\%$ of the values of any measurement should fall within one
standard deviation ($1 \sigma$) of the mean value ($\bar{t}$). Therefore $68\%$ of measured
values should be $\geq (t - \sigma_t)$ and $\leq (t + \sigma_t)$

\subsubsection{Percentage of values which fall in one standard deviation}

\subsubsection{Does the data match the statistical prediction}

\subsection{Values within two standard deviations} % 1.7
Approximately $95\%$ of the values of any measurement should fall within two
standard deviations ($2 \sigma$) of the mean value ($\bar{t}$). Therefore $95\%$ of measured
values should be $\geq (t - 2 \sigma_t)$ and $\leq (t + 2 \sigma_t)$

\subsubsection{Percentage of values which fall in two standard deviations}

\subsubsection{Does the data match the statistical prediction}

\subsection{Calculate the standard error of 5 different sets of measurements}


  $$SE = \frac{\sigma}{\sqrt{N}}$$
  $$SE = \frac{0.05224998188}{\sqrt{100}}$$
  $$SE = 0.005224998188$$
  $$SE \approx 0.0052$$

  \subsection{Calculate the standard error from your data set of 100 times}

  $$SE = \frac{\sigma_{100}}{\sqrt{N}}$$

  \section{Measuring the Time of a Dropped Shoe}

  \subsection{Predicted time $t_{thy}$} % 1.1

  \setcounter{subsection}{3} % Make next section start at 1.4
  \subsection{Mean average time for 100 measurements} % 1.4

  \subsection{Standard deviation for 100 measurements} % 1.5

  \subsection{Values within one standard deviation} % 1.6 

  \subsubsection{Percentage of values which fall in two standard deviations}

  \subsubsection{Does the data match the statistical prediction}

  \subsection{Calculate the standard error of 5 different sets of measurements}

  $$SE = \frac{\sigma_{5}}{\sqrt{N}}$$


\subsection{Calculate the standard error from your data set of 100 times}


  $$SE = \frac{\sigma_{100}}{\sqrt{N}}$$


\end{document}