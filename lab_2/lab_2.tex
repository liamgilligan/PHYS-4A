\documentclass[11pt, letterpaper, includehead]{article}

%%%%%%%%%%%%%%%%%%%%% Pre-document %%%%%%%%%%%%%%%%%%%%%
\usepackage{fancyhdr}  % Allow for headers
\usepackage{graphicx}  % Allow for figures 
\usepackage{float}     % Allow for figure inserted in specified location
\usepackage{xcolor}

\setlength{\parindent}{0pt} % Remove auto paragraph indents

% Get rid of those big ass margins
\usepackage[margin=1in]{geometry}

\begin{document}
  %%%%%%%%%%%%%%%%%%%%% Title Page %%%%%%%%%%%%%%%%%%%%%
  \begin{titlepage} 
    \begin{center}
      \Huge{\textbf{Lab 2}}\\
      \Huge{Experimental Uncertainty}
      \vfill
      \large{\textbf{Rectangle Repulsed Researchers}}\\
      \large{Julian Barossi, Liam Gilligan, Stephanie L'Heureux}\\
      \vspace{0.5cm}
      \normalsize
      \today
    \end{center}
  \end{titlepage}

  %%%%%%%%%%%%%%%%%%%%% TABLE OF CONTENTS %%%%%%%%%%%%%%%%%%%%%
  \tableofcontents
  \pagebreak % Move to next page


  % Add a nice fancy header
  \pagestyle{fancy}
  \fancyhead{}
  \fancyhead[C]{\textbf{Lab 2:} Experimental Uncertainty}

  \section{Measuring the Time of a Dropped Pencil} % 1
  
  \subsection{Predicted time $t_{thy}$ for the pencil to fall from rest $2.0m$} % 1.1
  
  \subsection{Recorded time for each trial} % 1.2

  \setcounter{subsection}{3} % Make next section start at 1.4
  \subsection{Mean average time ($\bar{t}$) for 100 measurements} % 1.4

  \subsection{Standard deviation ($\sigma$) for 100 measurements} % 1.5

  \subsection{Approximately $68\%$ of the values of any measurement should fall within one 
  standard deviation ($1 \sigma$) of the mean value ($\bar{t}$). Therefore $68\%$ of measured 
  values should be $\geq (t - \sigma_t)$ and $\leq (t + \sigma_t)$} % 1.6 

  \subsubsection{Percentage of values which fall in one standard deviation 
  ($1\sigma$) of the mean ($\bar{t}$)}

  \subsubsection{Does the randomly distributed data match up with the statsitical approximation
  that $68\%$ should fall within one standard deviation ($1\sigma$) of the mean ($\bar{t}$)}

  \subsection{Approximately $95\%$ of the values of any measurement should fall within two 
  standard deviations ($2 \sigma$) of the mean value ($\bar{t}$). Therefore $95\%$ of measured 
  values should be $\geq (t - 2 \sigma_t)$ and $\leq (t + 2 \sigma_t)$} % 1.7 

  \subsubsection{Percentage of values which fall in one standard deviation 
  ($1\sigma$) of the mean ($\bar{t}$)}

  \subsubsection{Does the randomly distributed data match up with the statsitical approximation
  that $68\%$ should fall within one standard deviation ($1\sigma$) of the mean ($\bar{t}$)}



\end{document}

