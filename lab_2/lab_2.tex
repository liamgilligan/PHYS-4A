\documentclass[11pt, letterpaper, includehead]{article}

%%%%%%%%%%%%%%%%%%%%% Pre-document %%%%%%%%%%%%%%%%%%%%%
\usepackage{fancyhdr}  % Allow for headers
\usepackage{graphicx}  % Allow for figures 
\usepackage{float}     % Allow for figure inserted in specified location
\usepackage{amsmath}   % Allow for aligned math
\usepackage{array}     % Allow for cell width manipulation
\usepackage[dvipsnames]{xcolor}    % Allow for colors + more colors!
\usepackage{amssymb} % Uhhhh what was this????


\setlength{\parindent}{0pt} % Remove auto paragraph indents

% Get rid of those big ass margins
\usepackage[margin=1in]{geometry}

% Table cell formatting
\setlength{\arrayrulewidth}{0.25mm}
\setlength{\tabcolsep}{11pt}
\renewcommand{\arraystretch}{1.2}

\begin{document}
  %%%%%%%%%%%%%%%%%%%%% Title Page %%%%%%%%%%%%%%%%%%%%%
  \begin{titlepage} 
    \begin{center}
      \Huge{\textbf{Lab 2}}\\
      \Huge{Experimental Uncertainty}
      \vfill
      \large{\textbf{Rectangle Repulsed Researchers}}\\
      \large{Julian Barossi, Liam Gilligan, Stephanie L'Heureux}\\
      \vspace{0.5cm}
      \normalsize
      \today
    \end{center}
  \end{titlepage}

  %%%%%%%%%%%%%%%%%%%%% TABLE OF CONTENTS %%%%%%%%%%%%%%%%%%%%%
  \tableofcontents
  \pagebreak % Move to next page


  % Add a nice fancy header
  \pagestyle{fancy}
  \fancyhead{}
  \fancyhead[C]{\textbf{Lab 2:} Experimental Uncertainty}

  \section{Measuring the Time of a Dropped Pencil} % 1
  
  \subsection{Predicted time $t_{thy}$} % 1.1
  The calculations below predict the time ($t_{thy}$) for the pencil to 
  fall from rest a distance of $2.0m$.
  
  \hspace*{0.25cm}
  \begin{tabular}{| c | c | c | c | c | c | l |} 
    \hline
    $y_0$ & $y$  & $v_{0_y}$ & $v_{y}$ & $a_y$ & $t$ & Equation \\ 
    \hline \hline
    \color{red}{$\times$} & \color{LimeGreen}{\checkmark} & \color{LimeGreen}{\checkmark} & \color{LimeGreen}{\checkmark} & \color{LimeGreen}{\checkmark} & \color{LimeGreen}{\checkmark} & $v_y = v_{y_0} + a_yt$ \\ 
    \hline
    \color{LimeGreen}{\checkmark} & \color{LimeGreen}{\checkmark} & \color{LimeGreen}{\checkmark} & \color{red}{$\times$} & \color{LimeGreen}{\checkmark} & \color{LimeGreen}{\checkmark} & $y = y_0 + v_{y_0}t + \frac{1}{2}a_xt^2$\\ 
    \hline
    \color{LimeGreen}{\checkmark} & \color{LimeGreen}{\checkmark} & \color{LimeGreen}{\checkmark} & \color{LimeGreen}{\checkmark} & \color{LimeGreen}{\checkmark} & \color{red}{$\times$} & $v_y^2 = v^2_{0_y} + 2a_y(x - x_0)$\\
    \hline
    \color{LimeGreen}{\checkmark} & \color{LimeGreen}{\checkmark} & \color{LimeGreen}{\checkmark} &  \color{LimeGreen}{\checkmark} &  \color{red}{$\times$} & \color{LimeGreen}{\checkmark} & $y - y_0 = \frac{1}{2}(v_{0_y} + v_y)t$\\
    \hline\hline
    $2.0m$ & $0.0m$ & $0.0m/s$ &  & $-9.8m/s^2$ & ? & $y = y_0 + v_{y_0}t + \frac{1}{2}a_xt^2$ \\
    \hline
  \end{tabular}

  \vspace{0.25cm}

  \renewcommand{\arraystretch}{1.75}
  \begin{tabular}{ m{4cm}  l } 
    $y = y_0 + v_{y_0}t + \frac{1}{2}a_yt^2$ & $y = 0m$ and $v_{0_y} = 0m/s$  \\
    $-y_0 = \frac{1}{2}a_yt^2$\\
    $t^2 = \frac{-2y_0}{a_y}$\\
    $t = \sqrt{\frac{-2y_0}{a}}$\\
    $t = \sqrt{\frac{-2(2.0m)}{-9.8m/s^2}}$\\
    $t = \sqrt{\frac{20}{49}}s$\\
    $t \approx 0.64s$\\
    $t_{thy}\approx 0.64s$
  \end{tabular} 

  % \begin{align*}
  %   v_0t + \frac{1}{2}at^2 &= x_f - x_0\\
  %   \frac{1}{2}at^2 &= -x_0\\
  %   t^2 &= \frac{-2x_0}{a}\\
  %   t &= \sqrt{\frac{-2x_0}{a}}\\
  %   t &= \sqrt{\frac{-2(2.0m)}{-9.8m/s^2}}\\
  %   t &= \sqrt{\frac{20}{49}}s\\
  %   t &\approx 0.64s
  % \end{align*}
  \renewcommand{\arraystretch}{1.2}
  \subsection{Recorded time} % 1.2 
    \begin{tabular}[H]{| m{2cm} | m{2cm} |}
      \hline
      \textbf{Trial} & \textbf{Time (s)} \\ [0.5ex]
      \hline \hline
        1 & 0.55 \\ 
         \hline
        2 & 0.69 \\ 
         \hline
        3 & 0.69 \\ 
         \hline
        4 & 0.65 \\ 
         \hline
        5 & 0.67 \\ 
         \hline
        6 & 0.58 \\ 
         \hline
        7 & 0.61 \\ 
         \hline
        8 & 0.64 \\ 
         \hline
        9 & 0.63 \\ 
         \hline
        10 & 0.68 \\ 
         \hline
        11 & 0.61 \\ 
         \hline
        12 & 0.63 \\ 
         \hline
        13 & 0.67 \\ 
         \hline
        14 & 0.64 \\ 
         \hline
        15 & 0.65 \\ 
         \hline
        16 & 0.68 \\ 
         \hline
        17 & 0.66 \\ 
         \hline
        18 & 0.66 \\ 
         \hline
        19 & 0.56 \\ 
         \hline
        20 & 0.66 \\ 
         \hline
  \end{tabular} 
 
  \setcounter{subsection}{3} % Make next section start at 1.4
  \subsection{Mean average time for 100 measurements} % 1.4

  \subsection{Standard deviation for 100 measurements} % 1.5

  \subsection{Values within one standard deviation} % 1.6 
  Approximately $68\%$ of the values of any measurement should fall within one 
  standard deviation ($1 \sigma$) of the mean value ($\bar{t}$). Therefore $68\%$ of measured 
  values should be $\geq (t - \sigma_t)$ and $\leq (t + \sigma_t)$

  \subsubsection{Percentage of values which fall in one standard deviation}

  \subsubsection{Does the data match the statistical prediction}

  \subsection{Values within two standard deviations} % 1.7
  Approximately $95\%$ of the values of any measurement should fall within two 
  standard deviations ($2 \sigma$) of the mean value ($\bar{t}$). Therefore $95\%$ of measured 
  values should be $\geq (t - 2 \sigma_t)$ and $\leq (t + 2 \sigma_t)$

  \subsubsection{Percentage of values which fall in two standard deviations}

  \subsubsection{Does the data match the statistical prediction}

  \subsection{Calculate the standard error of 5 different sets of measurements}

  \subsection{Calculate the standard error from your data set of 100 times}
  
\end{document}

